%=========================================================================
% (c) Michal Bidlo, Bohuslav K�ena, 2008

\chapter{FreeIPA}

FreeIPA (where IPA stands for Identity, Policy and Audit) is an open-source security management solution sponsored by Red Hat aimed primarily at Linux and Unix machines\cite{ipaWeb}.

The project itself combines a number of various existing open-source technologies to achieve the goal of providing centralized authentication and authorization, as well as storing important account information like users or group memberships.
FreeIPA also aims to provide easy management and setup of a domain controller which would otherwise be very difficult by using the same components on your own.

In this chapter I will briefly introduce some of the components FreeIPA uses and describe the architecture of the resulting FreeIPA server solution.

\section{Directory Server}
FreeIPA's directory service is the is built using the 389 Directory Server\cite{ldapWeb} and is used to store various information of all of FreeIPA's components.
It also plays a big role in authentication and authorization using Kerberos which will be presented in the next section.

The LDAP protocol\cite{ldapRFC} is used as a means of communication with the 389 DS and the data itself is stored in a Directory Information Tree (DIT) which is a tree-like data structure.

LDAP provides several operations to use with the server\cite{ldapRFC}:

\begin{itemize}
    \item \textbf{add, delete, modify:} These operations add, remove and modify the data contained in the DIT.
    \item \textbf{search, compare:} The search and compare operations are used in querying the DIT for specific information.
    \item \textbf{bind, unbind, abandon:} These operations can be used to authenticate to the directory, terminating the connection or abandoning a previously sent request entirely, respectively.
    \item \textbf{extended operations:} New operations that are not a part of the original protocol.
\end{itemize}

% TODO: ACLs

\section{Kerberos}
\section{DNS}
\section{Dogtag}
\section{FreeIPA Architecture}

\chapter{Active Directory}
\chapter{Analyze}
\chapter{Conclusion}

%=========================================================================
